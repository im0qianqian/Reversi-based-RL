% !TEX program = xelatex

\documentclass[10pt,aspectratio=169,mathserif]{beamer}		
%设置为 Beamer 文档类型,设置字体为 10pt,长宽比为16:9,数学字体为 serif 风格

%%%%-----导入宏包-----%%%%
\usepackage{ecnu}			%导入 CCNU 模板宏包
\usepackage{ctex}			 %导入 ctex 宏包,添加中文支持
\usepackage{amsmath,amsfonts,amssymb,bm}   %导入数学公式所需宏包
\usepackage{color}			 %字体颜色支持
\usepackage{graphicx,hyperref,url}	
%%%%%%%%%%%%%%%%%%	


\beamertemplateballitem		%设置 Beamer 主题

%%%%------------------------%%%%%
\catcode`\。=\active         %或者=13
\newcommand{。}{.}				
%将正文中的“。”号转换为“.”。
%%%%%%%%%%%%%%%%%%%%%

%%%%----首页信息设置----%%%%
\title[Design and Implementation of Othello Based on Reinforcement Learning]{基于强化学习的黑白棋的设计与实现}
\subtitle{Design and Implementation of Othello Based on Reinforcement Learning}			
%%%%----标题设置


\author[by im0qianqian]{
  Zhao Qian \\\medskip
  {\small {Adviser: Jerry Wang}} \\
}
%%%%----个人信息设置

\institute[YTU]{
  Yantai University \\
  School of Computer and Control Engineering}
%%%%----机构信息

\date[2019.06.01111]{
  2019.06.01}
%%%%----日期信息

\begin{document}

\begin{frame}
\titlepage
\end{frame}				%生成标题页

\section{Outline}
\begin{frame}
\frametitle{Outline1}
\tableofcontents
\end{frame}				%生成提纲页

\section{Introduction}
\begin{frame}
  \frametitle{Backgrounds}
  \begin{itemize}
    \item {1111111111111}
    \item {2222222222222}
  \end{itemize}
  \begin{enumerate}
    \item {3333333}
    \item {44444444}
  \end{enumerate}
  \begin{description}
    \item[First Item] Description of first item
    \item[Second Item] Description of second item
    \item[Third Item] Description of third item
    \item[Forth Item] Description of forth item
   \end{description}
\end{frame}

\section{Photo}
\begin{frame}
  \frametitle{My Photo}
    \begin{figure}[htbp]
        \centering
        \includegraphics[width=0.3\textwidth]{4.jpg}
        \caption{hahahaha...}\label{fig:digit}
    \end{figure}
\end{frame}

\section{Conclusion}
\begin{frame}
  \frametitle{Emmm...}
  \noindent\textbf{Sequence Tagging Loss\\}
    \[{\mathcal{L}_p} =  - \sum\limits_{i = 1}^S {\sum\limits_{j = 1}^N {{p_{i,j}}\log ({{\hat p}_{i,j}})} } \]
  \noindent\textbf{Language Classifier Loss\\}
   \[{\mathcal{L}_a} =  - \sum\limits_{i = 1}^S {{l_i}\log ({{\hat l}_i})}\]
  \noindent\textbf{Bidirectional Language Model Loss\\}
   \[{\mathcal{L}_l} =  - \sum\limits_{i = 1}^S {\sum\limits_{j = 1}^N {\log (P({w_{j + 1}}|{f_j})) + \log (P({w_{j - 1}}|{b_j}))} }\]
\end{frame}

\section{References}
\begin{frame}{References}
\begin{thebibliography}{99}
\bibitem{zhao1} Xuezhe Ma and Eduard Hovy. (2016).\\
{\bf End-to-end Sequence Labeling via Bi-directional LSTM-CNNs-CRF.\\}
In Proceedings of the 54th Annual Meeting of the Association for Computational Linguistics, pages 1064–1074, Berlin, Germany, August 7-12, 2016.
\bibitem{zhao2} Marek Rei. (2017).\\
{\bf Semi-supervised Multitask Learning for Sequence Labeling.\\}
In Proceedings of the 55th Annual Meeting of the Association for Computational Linguistics, pages 2121–2130, Vancouver, Canada, July 30 - August 4, 2017.
\end{thebibliography}
\end{frame}

\end{document}
